\documentclass[slidetop,11pt]{beamer}
%
% Ces deux lignes {\`a} d{\'e}commenter pour sortir 
% le texte en classe article
% \documentclass[class=article,11pt,a4paper]{beamer}
% \usepackage{beamerbasearticle}

% Packages pour les fran\c{c}ais
%
\usepackage[T1]{fontenc} 
\usepackage[latin1]{inputenc}
\usepackage[frenchb]{babel}
% pour un pdf lisible {\`a} l'{\'e}cran si on ne dispose pas 
% des fontes cmsuper ou lmodern
%\usepackage{lmodern}
\usepackage{aeguill}

% Pour afficher le pdf en plein ecran
% (comment{\'e} pour imprimer les transparents et pour les tests)
%\hypersetup{pdfpagemode=FullScreen}

% ------------------------------------------------
%-----------   styles pour beamer   --------------
% ------------------------------------------------
%
% ------------- Choix des couleurs ---------------

%% %% %% %% %% COULEURS PERSOS
\definecolor{verylightblack}{rgb}{0.25,0.25,0.25}
\definecolor{lightblack}{rgb}{0.1,0.1,0.1}
\definecolor{black}{rgb}{0,0,0}


\definecolor{verylightgrey}{rgb}{0.8,0.8,0.8}
\definecolor{lightgrey}{rgb}{0.6,0.6,0.6}
\definecolor{grey}{rgb}{0.4,0.4,0.4}

\definecolor{verylightgray}{gray}{0.80}
\definecolor{lightgray}{gray}{0.6}
\definecolor{gray}{gray}{0.4}

\definecolor{verylightblue}{rgb}{0.9,0.9,1.0}
\definecolor{lightblue}{rgb}{0.75,0.75,1.0}
\definecolor{blue}{rgb}{0.5,0.5,1.0}
\definecolor{darkblue}{rgb}{0.0,0.0,0.5}

\definecolor{verylightred}{rgb}{1.0,0.9,0.9}
\definecolor{lightred}{rgb}{1.0,0.75,0.75}
\definecolor{red}{rgb}{1.0,0.5,0.5}

\definecolor{verylightgreen}{rgb}{0.9,1.0,0.9}
\definecolor{lightgreen}{rgb}{0.75,1.0,0.75}
\definecolor{green}{rgb}{0.5,1.0,0.5}


\definecolor{verylightcolorGREY}{gray}{0.80}
\definecolor{lightcolorGREY}{gray}{0.60}

\definecolor{verylightcolorRED}{rgb}{1.0,0.9,0.9}
\definecolor{lightcolorRED}{rgb}{1.0,0.75,0.75}

\definecolor{verylightcolorUN}{rgb}{1.0,0.75,1.0}
\definecolor{lightcolorUN}{rgb}{1.0,0.50,1.0}

\definecolor{verylightcolorDE}{rgb}{0.75,1.0,1.0}
\definecolor{lightcolorDE}{rgb}{0.50,1.0,1.0}

\definecolor{verylightcolorTR}{rgb}{1.0,1.0,0.75}
\definecolor{lightcolorTR}{rgb}{1.0,1.0,0.50}


% Red{\'e}finit la couleur de fond pour imprimer sur transparents
%\xdefinecolor{fondtexte}{rgb}{1,1,1}     % blanc

% commande differente pour les couleurs nomm{\'e}es - de base
%\colorlet{coultexte}{black} 

% -------------- Fioritures de style -------------
% Fait afficher l'ensemble du frame 
% en peu lisible (gris clair) d{\`e}s l'ouverture
\beamertemplatetransparentcovered

% Supprimer les icones de navigation (pour les transparents)
%\setbeamertemplate{navigation symbols}{}

% Mettre les icones de navigation en mode vertical (pour projection)
%\setbeamertemplate{navigation symbols}[vertical]

% ------------ Choix des th{\`e}mes ------------------
\usecolortheme{default} % gabywald
% \usecolortheme{orchid}

\setbeamercolor{title}{fg=black, bg=red!40}
\setbeamercolor{block title}{fg=black, bg=red!40}
\setbeamercolor{structure}{fg=black, bg=red!40}
\setbeamercolor{block title}{fg=black, bg=lightred!40}
\setbeamercolor{substructure}{fg=black, bg=verylightred!40}

\setbeamercolor{block title GREY}{fg=black, bg=lightcolorGREY!40}
\setbeamercolor{substructureGREY}{fg=black, bg=verylightcolorGREY!40}

\setbeamercolor{block title RED}{fg=black, bg=lightcolorRED!40}
\setbeamercolor{substructureRED}{fg=black, bg=verylightcolorRED!40}

\setbeamercolor{block title UN}{fg=black, bg=lightcolorUN!40}
\setbeamercolor{substructureUN}{fg=black, bg=verylightcolorUN!40}

\setbeamercolor{block title DE}{fg=black, bg=lightcolorDE!40}
\setbeamercolor{substructureDE}{fg=black, bg=verylightcolorDE!40}

\setbeamercolor{block title TR}{fg=black, bg=lightcolorTR!40}
\setbeamercolor{substructureTR}{fg=black, bg=verylightcolorTR!40}

% \setbeamercolor{block title}{fg=black, bg=lightblue!40}
% \setbeamercolor{block body}{...}
% \setbeamercolor{block title example}{...}
% \setbeamercolor{block body example}{...}
% \setbeamercolor{block title alerted}{...}
% \setbeamercolor{block body alerted}{...}

% \useoutertheme[left]{sidebar}
% \setbeamersize{sidebar left width=3.0cm \tableofcontents[hideothersubsections] }
% \setbeamercolor{sidebar left}{fg=green,bg=lightgreen}
% \setbeamercolor{title in sidebar}{parent=title}

\setbeamercolor*{sidebar}{fg=lightblack,bg=lightblue!75!white}

\setbeamercolor*{palette sidebar primary}{fg=darkblue!50!lightgrey}
\setbeamercolor*{palette sidebar secondary}{fg=black} % darkblue!10!black
\setbeamercolor*{palette sidebar tertiary}{fg=darkblue!50!lightgrey}
\setbeamercolor*{palette sidebar quaternary}{fg=black} % darkblue!10!black

% \setbeamercolor{subsubsection in sidebar}{hideallsubsections}
% \setbeamercolor{subsubsection in sidebar shaded}{hideallsubsections}


%------------ fin style beamer -------------------

% Faire appara{\^i}tre un sommaire avant chaque section
% \AtBeginSection[]{
%   \begin{frame}
%   \frametitle{Plan}
%   \medskip
%   %%% affiche en d{\'e}but de chaque section, les noms de sections et
%   %%% noms de sous-sections de la section en cours.
%   \small \tableofcontents[currentsection, hideothersubsections]
%   \end{frame} 
% }

% ----------- Contenu de la page de titre --------
\title{Cyber Fizz Buzz}
\subtitle{Tests de programmation et recrutement en informatique} 
\author{Gabriel Chandesris}
\institute{\emph{to be defined}}
\date{\today} %% \date{03 Janvier 2018}
%% \logo{\includegraphics[height=0.5cm]{img/logo_glider.png}}

% ----------- Nom des diff{\'e}rentes parties --------

%% Mis au fur et {\`a} mesure...

\def\moreInFrameTitleLeftt{\includegraphics[height=0.5cm]{img/ligueludique-0.png}~~~~~}
\def\moreInFrameTitleRight{~~~~~\includegraphics[height=0.5cm]{img/logo_glider.png}}

% ------------------------------------------------
% -------------   D{\'e}but document   ---------------
% ------------------------------------------------
\begin{document}
%--------- {\'e}criture de la page de titre ----------
% avec la commande frame simplifi{\'e}e
\frame[plain]{\titlepage } 
%

%------------------ Sommaire ---------------

\begin{frame}
	\frametitle{Sommaire}
	\small \tableofcontents[hideallsubsections]
\end{frame} 

\section{Cyber Fizz Buzz}
\begin{frame}
	\frametitle{Cyber Fizz Buzz}
	\tableofcontents[sections=1,currentsection,subsectionstyle=show/shaded/hide] %% sectionstyle=hide/hide,
\end{frame} 

\subsection{Fizz Buzz K{\'e}zako ?}
\begin{frame}
	\frametitle{Fizz Buzz K{\'e}zako ?}
	\tableofcontents[sections=1,currentsection,subsectionstyle=show/shaded/hide]
\end{frame} 

\subsubsection{Cyber Fizz Buzz : tentative de d{\'e}finition}
\begin{frame}
	\frametitle{Cyber Fizz Buzz : tentative de d{\'e}finition}
	
	"{\'E}valuation d'un candidat lors d'un entretien technique. "~\\
	
	\begin{itemize}
		\item {\'E}valuer les connaissances et r{\'e}alisations ; 
		\item {\'E}valuer les interactions (client, coll{\`e}gues, ...) ; 
		\item Conna{\^i}tre le fonctionneemnt et la fa\c{c}on de r{\'e}fl{\'e}chir du candidat ; 
		\item Ouverture d'esprit du candidat (critique, construction, positionnement) ; 
		\item Connaissances de technique de programmation et algorithmique ; 
	\end{itemize}~\\
	
	Note : Pourquoi le CyberFizzBuzz ? Formations non homog{\`e}nes (m{\^e}me au sein d'un m{\^e}me cursus). 
\end{frame}

\subsubsection{Cyber Fizz Buzz : concepts}
\begin{frame}
	\frametitle{Cyber Fizz Buzz : concepts}
	\begin{itemize}
		\item element 1
		\item element 2
		\item element 3
	\end{itemize}
\end{frame} 

\subsubsection{Cyber Fizz Buzz : int{\'e}r{\^e}ts}
\begin{frame}
	\frametitle{Cyber Fizz Buzz : int{\'e}r{\^e}ts}
	\begin{itemize}
		\item element 1
		\item element 2
		\item element 3
	\end{itemize}
\end{frame} 

\subsection{CyberFizzBuzz : exemples}
\begin{frame}
	\frametitle{CyberFizzBuzz : exemples}
	\tableofcontents[sections=1,currentsection,subsectionstyle=show/shaded/hide]
\end{frame} 

\subsubsection{CyberFizzBuzz : exemples (1)}
\begin{frame}
	\frametitle{CyberFizzBuzz : exemples (1)}
	\begin{itemize}
		\item {\'E}crire une suite de nombres (for, stream, lambdas)...
		\item Op{\'e}rateur modulo (\%) ; 
		\item Classes, Design Patterns (concepts et applications) ; 
		\item Abstractions et Interfaces de POO ; 
		\item Analyser logique d'un algorithme : entier ou r{\'e}el le plus proche de 0 ; 
		\item Recherche dans une matrice ; 
		\item Parcours d'arbres (classification, valeurs...) ; 
		\item ... 
	\end{itemize}
\end{frame} 

\subsubsection{CyberFizzBuzz : exemples (2)}
\begin{frame}
	\frametitle{CyberFizzBuzz : exemples (2)}
	\begin{itemize}
		\item {\`A} partir d'un {\'e}nonc{\'e} de base, consturction tests et codes avec commentaires et explications ; 
		\item L'objectif principal n'est pas de r{\'e}aliser l'ensemble du projet d{\'e}crit dans l'{\'e}nonc{\'e} ; 
		\item Exercice volontairement trop long par rapport au temps imparti ; 
		\item Analyse du cheminement (architecture et approche utilis{\'e}es par le candidat), plut{\^o}t que le r{\'e}sultat ;
		\item Questions compl{\'e}mentaires {\`a} partir de l'{\'e}nonc{\'e} (interaction client) ; 
		\item Propositions  et impl{\'e}mentations de structures, tests unitaires, tests d'int{\'e}gration ;
		\item Design Patterns, Optimisations, Approches utilis{\'e}es ; ... 
	\end{itemize}
\end{frame} 

\section{Section !}
\begin{frame}
	\frametitle{Section !}
	\tableofcontents[sections=2,currentsection,subsectionstyle=show/shaded/hide] %% sectionstyle=hide/hide,
\end{frame} 

\subsection{Sous-section !}
\begin{frame}
	\frametitle{Sous-section ! }
	\tableofcontents[sections=2,currentsection,subsectionstyle=show/shaded/hide]
\end{frame} 

\subsubsection{Sous sous section}
\begin{frame}
	\frametitle{Sous sous section  (1)}
	\begin{itemize}
		\item element 1
		\item element 2
		\item element 3
	\end{itemize}
\end{frame} 

\section{Section !}
\begin{frame}
	\frametitle{Section !}
	\tableofcontents[sections=3,currentsection,subsectionstyle=show/shaded/hide] %% sectionstyle=hide/hide,
\end{frame} 

\subsection{Sous-section !}
\begin{frame}
	\frametitle{Sous-section ! }
	\tableofcontents[sections=3,currentsection,subsectionstyle=show/shaded/hide]
\end{frame} 

\subsubsection{Sous sous section}
\begin{frame}
	\frametitle{Sous sous section  (1)}
	\begin{itemize}
		\item element 1
		\item element 2
		\item element 3
	\end{itemize}
\end{frame} 

\def\sectionPartBibliographie{Bibliographie / Mediagraphie}
\section{\sectionPartBibliographie}
\begin{frame}
	\frametitle{\sectionPartBibliographie}
	\nocite{*}
	%toutes references biblio : 6 lettres + 2 chiffres
	\bibliography{presentationCyberFizzBuzz}
	% \bibliographystyle{frplain} % plain or frplain
	\bibliographystyle{plain}
\end{frame}

\end{document}
